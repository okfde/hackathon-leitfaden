% \%setchapterpreamble[u]{\margintoc}
%\chapter{Einleitung}
%\labch{intro}


\setchapterimage[6cm]{donut.jpg}
\setchapterpreamble[u]{\margintoc}
\chapter{Einleitung}
\labch{layout}


Stellen Sie sich vor, es gibt einen Ort, an dem viele verschiedene Menschen zusammenkommen.

Zugegeben – in Zeiten von Corona nicht ganz einfach. Doch auch im Internet kann man Orte schaffen, an denen Menschen zusammenkommen und gemeinsam weiter denken.

Ein mögliches Format, das Austausch und Partizipation ermöglicht, ist der Hackathon. 

Dieser Leitfaden gibt Anregungen, wie ein Hackathon gelingen kann.

\newpage

\section{Was ist ein Hackathon?}
\labsec{does}

Im Alltag tauchen immer wieder Probleme auf, die nicht nebenbei lösbar sind: Sie betreffen sehr viele Menschen, brauchen neue Infrastruktur oder erfordern Expertise, die in der eigenen Institution einfach nicht vorhanden ist.

Bei einem Hackathon kommen verschiedene Menschen für einen bestimmten Zeitraum zusammen, um sich intensiv mit einem Themenfeld zu beschäftigen. Das Format kommt aus der Entwicklerszene und im Kern geht es oft darum, eine - meist technische - Lösung für ein Problem zu finden – oder kurz: zu programmieren.

% \marginnote[2mm]{The audacious users might feel tempted to edit some of 	these packages. I'd be immensely happy if they sent me examples of what 	they have been able to do!}

\subsection*{Mehr als nur ein Prototyp}

Tatsächlich hat sich in der Praxis allerdings herausgestellt, dass diese "Lösungen" häufig nicht unbedingt in der Entwicklung eines technischen Prototypen liegen, sondern viel weitreichendere Fortschritte erzielen können. 

\begin{marginfigure}[-5.5cm]
	\includegraphics{monalisa.png}
	%	\caption[The Mona Lisa]{The Mona Lisa.\\ 
	% 		\url{https://commons.wikimedia.org/wiki/File:Mona_Lisa,_by_Leonardo_da_Vinci,_from_C2RMF_retouched.jpg}}
	\labfig{marginmonalisa}
\end{marginfigure}

Die Empfehlungen hier zielen auf kommunale und andere öffentliche Institutionen ab. Hackathons im wirtschaftlichen Zusammenhang haben i.d.R. etwas andere Schwerpunkte.

Hackathons eignen sich hervorragend, um

\begin{itemize}
	\item \textbf{persönliche Beziehungen aufzubauen}, \newline beispielsweise zur lokalen Gruppe von Aktiven im Civic-Tech-Bereich. 
	\vspace{0.5cm}
	\item \textbf{interessierte Menschen zusammenzubringen} \newline und so nicht nur persönliche Vernetzung vor Ort, sondern auch Wissensaustausch zu ermöglichen.
	\vspace{0.5cm}
	\item \textbf{Bestehende Ansatzpunkte und Systeme identifizieren}, \newline die lokal in der Verwaltung umgesetzt werden können. Häufig muss das Rad nicht neu erfunden werden, sondern es gibt in anderen Kontexten oder vielleicht sogar anderen Kommunen bereits bestehende und gut funktionierende Systeme. Während der Corona-Pandemie haben beispielsweise die Berliner Bäder begonnen, ihre Schwimmzeigen über das Open-Source-System pretix zu vergeben. Dieses System hatte sich Jahre zuvor insbesondere im Veranstaltungsbereich bewährt.
	\vspace{0.5cm}
	\item \textbf{Schwachstellen der kommunalen IT-Struktur erkennen}, \newline an denen die Umsetzung solcher Ideen bislang scheitert
\end{itemize}


\subsection*{Ein Format – viele Möglichkeiten}

In den meisten Fällen dauert ein Hackathon zwei oder drei Tage. In dieser Zeit wird häufig in festen Teams gearbeitet – mindestens genauso wichtig ist allerdings die Zeit am Buffet, in kurzen Input-Präsentationen zwischendurch oder in Workshops, bei den die Teilnehmenden neue Techniken erproben oder Fertigkeiten erlernen können.
Es gibt auch Hackathons, die über mehrere Wochen oder gar Monate laufen. In bestimmten Fällen kann es vorteilhaft sein, z.B. wenn online und international zusammengearbeitet wird, oder wenn die Fragestellung / Fokusthema sehr komplex ist. Jedenfalls könnte die Bezeichnung Hackathon hier missverständlich sein.


Eine große Rolle spielt, ob es sich um eine Präsenz- oder eine Onlineveranstaltung handelt. Auch Kombinationen aus den beiden Formen sind denkbar.

\section{Was ein Hackathon nicht ist}


Bei der Bewerbung von Hackathons wird häufig das Ziel propagiert, Prototypen, Produkte oder lauffähige Software zu erstellen. Die Praxis zeigt jedoch: Die meisten der Prototypen sind bereits vorhanden, teilweise sogar in besseren Versionen.

In der Kürze der Zeit ist es auch für eine ausgebildete Fachjury nur schwer möglich, die Qualität der präsentierten "Produkte" angemessen abzuschätzen. Insbesondere wenn ein Preisgeld ausgelobt wird, arbeiten Teilnehmende teilweise mehr an der abschließenden Präsentation, um eine gute Bewertung der Jury zu bekommen, statt inhaltlich zu arbeiten. So sind am Ende nicht nur die Teilnehmenden frustriert, sondern arbeiten auch in künstlich abgegrenzten Teams gegen- statt miteinander.

Manchmal bringen Teilnehmende bereits vorbereitete eigene Lösungen, um dafür oder andere eigene Inhalte Werbung zu machen. Das ist nicht immer im Sinne der ausrichtenden Institution und führt zumindest in einer Wettbewerb-Konstellation zu Verzerrungen.

\begin{kaobox}Die größten Probleme lassen sich nicht durch Prototypen lösen, denn in den meisten Fällen haben wir kein Erkenntnis-, sondern ein Umsetzungsproblem.
\end{kaobox}

An welcher Stelle ein Wettbewerb sinnvoll wäre, muss natürlich im Einzelfall enschieden werden. Auch wenn Konkurrenz bekanntlich "das Geschäft belebt", kann darunter wertvolle Kollaboration und der Beziehungsaufbau während des Hackathons, aber auch besonders die Offenheit und die Weiterverwendbarkeit der Ergebnisse leiden.

Geht es also darum, ein ganz konkretes Problem zu lösen, sollte eine gründliche Recherche betrieben und mögliche Lösungen auf anderen Wegen gesucht werden. Bei einem Hackathon können allerdings genausogut neue Impulse entstehen, oder eine Vorstellung, wo man nach weiteren Lösungen suchen könnte. Die Erwartung der fertig konzeptionierten, weltbewegenden neuen Idee wird jedenfalls sehr selten erfüllt.

Denn häufig liegt das Problem gar nicht in fehlenden Lösungen, sondern in der fehlenden Umsetzung – etwa, weil es an IT-Infrastruktur wie Servern oder Schnittstellen mangelt oder die nötige Datenbasis nicht vorhanden ist.


\subsection*{Werbung statt Inhaltliche Arbeit}

Öffentlichkeitsarbeit steht oft bei Hackathons von öffentlichen Institutionen im Vordergrund. Auch wenn die Kommunikation nach außen notwendig und wichtig ist, sollte das Image des Veranstalters nicht die primäre Motivation für die Durchführung sein. Eine Diskrepanz zwischen einer (im Sinne von PR oft positivierten) Außendarstellung und dem gefühlten Erfolg des Hackathons wird mindestens für die Teilnehmenden bemerkbar sein und zur Frustration führen.

\section{Am Anfang war der Hackathon}

\setchapterimage[6cm]{donut.jpg}
% \setcounter{margintocdepth}{1}
% \setchapterpreamble[u]{\margintoc}
\chapter{Der Weg zu einem gelungenen Hackathon}

Ein Hackathon an sich reicht nicht aus. Ein Hackathon 

% \section{\textsc{Teilziel 1:} \newline \color{Cerulean} Persönliche Beziehungen aufbauen}
\section{Persönliche Beziehungen aufbauen}

* Interesse an Offenen Daten
* Nachhaltigkeit wichtig: Was bleibt übrig vom Hackathon? – aber gleichzeitig auch: Was ist überhaupt leistbar?
* gibt es eine Kommunikationsinfrastruktur für vor, während und nach dem Hackathon?

\subsection{Informelle Kommunikation ermöglichen}

Wenn ein Hackathon als Präsenzveranstaltung stattfinden kann, ist dies leichter zu ermöglichen als bei einem reinen Online-Event. Für den Aufbau von Beziehungen spielt auch Spaß an gemeinsamer Arbeit sowie Raum für informelle Kommunikation eine Rolle. Wird ein Hackathon von öffentlichen Institutionen ausgerichtet und ist diese Ausrichtung z.B. aufgrund einer Förderung an Vorgaben gebunden, können Erwartungen / Pflichten seitens der Veranstalter die informelle kreative Atmosphäre be- oder verhindern.

\section{Interessierte Menschen zusammenbringen}

* alle einladen
* Code of Conduct
* sinnvolle Zeiten
* auch "Vorbeischauen" möglich?

(* Hinweis, dass datenschutzfreundliche Tools besonders gut bei der anvisierten Zielgruppe ankommen?)

\subsection{Unterschiedliche Motivationen berücksichtigen}

Sehr wertvoll ist im Sinne von öffentlichen Institutionen die folgende Brücke: einerseits die Expertise extrerner Teilnehmenden, z.B. in Technologie, Design oder anderen Spezialfeldern, andererseits die Expertise, Einblicke und Erfahrungen aus dem "inneren" der Institution, idealerweise aller Hierarchieebenen. Beide Gruppen haben allerdings unterschiedeliche Motivationen, an einem Hackathon teilzunehmen, dies sollte bei der Organisation berücksichtigt werden.

\subsection{Weiterenwicklung}


\section{Ansatzpunkte und Systeme identifzieren, die lokal in der Verwaltung umgesetzt werden können}

\subsection{Fachpersonal der Verwaltungen einbinden}
Grundsätzlich gilt: Je konkreter die Probleme sind, desto eher findet sich eine Lösung. Und wer könnte die eigenen Probleme so anschaulich und fachlich korrekt vorstellen wie die Mitarbeitenden selbst?

\subsection{Dokumentation}
Eine umfangreiche und strukturierte **offene** Dokumentation der Veranstaltung und vor allem der geleisteten Recherche und der (Projekt-)Arbeit live oder zeitnah gewährleisten. Beide Formen der Dokumentation erfordern ein Konzept, eine Struktur und die benötigten Mittel.

\section{Schwachpunkte in der kommunalen IT-Struktur erkennen}


\chapter{Klippen umschiffen – Was Sie dringend vermeiden sollten}

* Wettbewerb
* zu komplexe Teilnahmebedingungen
* exklusive Datenbereitstellung → https://codefor.de/blog/unvereinbarkeitserkl%C3%A4rung-offene-daten-und-ndas/





% \marginnote[2mm]{The audacious users might feel tempted to edit some of these packages. I'd be immensely happy if they sent me examples of what they have been able to do!}

